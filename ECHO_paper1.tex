\documentclass[preprint2]{aastex}

\usepackage{url}
\usepackage{multirow}
\usepackage{amsmath}
\usepackage{xcolor}
\usepackage{mathrsfs}
%\citestyle{aa}

%\bibliographystyle{apj_w_etal}

\newcommand{\etal}{{et al.\/}}
\newcommand{\Prob}{\mathtt{P}}
\newcommand{\logL}{\log\mathcal{L}}
\newcommand{\unit}[1]{\footnotesize #1}
\newcommand{\PAPER}{\mathrm{PAPER}}
\newcommand{\hMpci}{h\ {\rm Mpc}^{-1}}

\newcommand{\Nx}{$N_x$}
\newcommand{\MminX}{$M_{minX}$}
\newcommand{\alphaX}{$\alpha_X$}
\newcommand{\xray}{X-ray}

\newcommand{\HI}{HI}
%%define graphics path to search for images
%\graphicspath{{./data/}}


\definecolor{orange}{RGB}{255,127,0}

\tabletypesize{\scriptsize}

	% End definitions

%\slugcomment{DRAFT: \today}

\shorttitle{ECHO}
\shortauthors{Jacobs et al.}

\begin{document}


\title{The External Calibrator for Hydrogen Arrays}
\author{
Daniel C. Jacobs\altaffilmark{1},
Jacob Burba\altaffilmark{1},
Judd Bowman\altaffilmark{1},
Lauren Turner\altaffilmark{1}
Kali Johnson\altaffilmark{1}}
\altaffiltext{1}{School of Earth and Space Exploration, Arizona State U., Tempe, AZ, 85287}



\begin{abstract}
We describe the External Calibrator for Hydrogen Observatories (ECHO) 
\end{abstract}



\section{Introduction}\label{sec:intro}

\section{Design and Method}
\section{Calibration}
\subsection{Maps}
\subsection{Comparison to Orbcomm ratio maps}
\section{MWA Tile}
\subsection{data}
\subsection{systematic variation}
\subsection{comparison to orbcomm maps}
\section{Conclusion}


\section{Acknowledgments}{

ECHO is supported by a grant from the National Science Foundation (NSF; award number 1407646). D.C.J would like to acknowledge NSF support  under award 1401708.
Thanks to the National Radio Astronomy Observatory, Green Bank and to Embry Riddle Aeronautical Observatory for site cooperatively supporting site operations.
}
\bibliographystyle{apj}
\bibliography{library}



\end{document}
